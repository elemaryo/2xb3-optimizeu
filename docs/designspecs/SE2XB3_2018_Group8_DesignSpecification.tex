\documentclass[12pt]{article}

\usepackage{fullpage}
\usepackage{booktabs}
\usepackage{tabularx} % Resizable tabular environment

\pagestyle{plain}
\pagenumbering{arabic}

\newcounter{stepnum}

%%%%%%%%%%%%%%%
% The environment below is used for text file extension Verbatim inclusion
%%%%%%%%%%%%%%%
\usepackage[dvipsnames]{xcolor}

\usepackage{fancyvrb}

% redefine \VerbatimInput
\RecustomVerbatimCommand{\VerbatimInput}{VerbatimInput}%
{fontsize=\footnotesize,
 %
 frame=lines,  % top and bottom rule only
 framesep=2em, % separation between frame and text
 rulecolor=\color{Gray},
 %
 label=\fbox{\color{Black}Revision History},
 labelposition=topline,
 %
 commandchars=\|\(\), % escape character and argument delimiters for
                      % commands within the verbatim
 commentchar=*        % comment character
}

\begin{document}

\begin{titlepage} % Suppresses headers and footers on the title page

	\centering % Centre everything on the title page
	
	\scshape % Use small caps for all text on the title page
	
	\vspace*{\baselineskip} % White space at the top of the page
	
	%------------------------------------------------
	%	Title
	%------------------------------------------------
	
	\rule{\textwidth}{1.6pt}\vspace*{-\baselineskip}\vspace*{2pt} % Thick horizontal rule
	\rule{\textwidth}{0.4pt} % Thin horizontal rule
	
	\vspace{0.75\baselineskip} % Whitespace above the title
	
	{\LARGE SE2XB3 OptimizeU V1.0 \\ Design Specification\\} % Title
	
	\vspace{0.75\baselineskip} % Whitespace below the title
	
	\rule{\textwidth}{0.4pt}\vspace*{-\baselineskip}\vspace{3.2pt} % Thin horizontal rule
	\rule{\textwidth}{1.6pt} % Thick horizontal rule
	
	\vspace{2\baselineskip} % Whitespace after the title block
	
	%------------------------------------------------
	%	Subtitle
	%------------------------------------------------
	
	Final Project for SFWRENG 2XB3:Software Engineering Practice and Experience: 
	Binding Theory to Practice
	
	\vspace*{3\baselineskip} % Whitespace under the subtitle
	
	%------------------------------------------------
	%	Group members
	%------------------------------------------------
	
	Group 8\\ Members:\\
	
	\vspace{0.5\baselineskip}
	
	{\scshape\Large Lucas Matthew Dutton \\ Michael Le \\ Saad Khan \\ Omar Elemary\\}
	
	\vspace{0.5\baselineskip} 
	
	\textit{McMaster University \\ Department of Computing and Software} 
	
	\vfill
	
	%------------------------------------------------
	%	Date
	%------------------------------------------------
	
	Created and Edited on: April 1, 2018 

\end{titlepage}

\tableofcontents

\newpage

\section{Revisions}

\subsection{Revision History}

\VerbatimInput{log.txt}

\newpage

\subsection{Team Details}

\begin{table}[h]
\begin{tabular}{| l | l | l |}
\hline
Name & Student Number & Roles \\ 
\hline
Michael Le & 400070369 & Group Leader, Graphics Implementation\\ 
\hline
Lucas Dutton & 400052930 & Log Admin, Algorithms Implementation\\ 
\hline
Saad Khan & 400085498 & Documentation Designer, Server Researcher\\ 
\hline
Omar Elemary & 400100169 & Algorithms Researcher, Log Assistant\\ 
\hline
\end{tabular}
\end{table}

\subsection{Attestation and Consent}

\textit{By virtue of submitting this document we electronically sign
and date that the work being submitted by all the individuals in the 
group is their exclusive work as a group and we consent to make
available the application developed through SE2XB3 project, the 
reports, presentations and assignments (not including my name or 
student number) for future teaching purposes.}

\newpage

\section{Contributions}
Note: Member roles are outlines in \textbf{Revisions: Team Details}

\begin{table}[h]
\begin{tabularx}{\textwidth}{|l|X|X|}
\hline
Name & Contributions & Comments \\
\hline
Lucas Dutton & Implemented Kruskals, K-Means and Associated ADT
               (Abstract Data Type)
             & Responsible for implementing and updating algorithms
               with any project changes\\
\cline{2-3}
~ & Project Log and Meeting Minutes admin 
  & Responsible for monitoring and updating Project Logs\\
\cline{2-3}
~ & Documentation editor
  & Responsible for creating and sharing documents such as 
    Requirements and Design Specifications\\
\hline
Michael Le & Implemented graphical representation of outputs
           & Responsible for presenting data in a graphical
             environment in Java\\
\cline{2-3}
~ & Implemented graphical ADTs 
  & Cluster and Cord ADTs made for interaction between algorithms
    and graphics\\
\hline
Saad Khan & Presentation Facilitator 
          & Responsible for organizing presentation slides\\
\cline{2-3}
~ & Graphics creator for documents 
  & Responsible for creating diagrams for documents, e.g. UML\\
\hline
Omar Elemary & Meeting Minutes upkeep
             & Responsible for creating and updating project logs\\
\cline{2-3}
~ & Algorithms research
  & Responsible for researching algorithms in the project proposal\\
\hline
\end{tabularx}
\end{table}

\newpage

\section{Executive Summary}
OptimizeU is intended to be a portable app that addresses the problem of finding an Uber pickup
on busy nights, as well as optimizing driving routes of Uber drivers. The project leverages a 
dataset of twenty million Uber pick ups in New York City to help drivers locate the
busiest hotspots in the city, routing more cars to denser locations which allows more users
to find access to an Uber vehicle in a faster time. The application uses various algorithms
related to machine learning and graphing, as well as pre-processing the dataset with searching
and sorting algorithms to generate a pick-up density map of the city. The map will contain
heat clusters related to density and distances between each cluster to inform the driver of
optimal routes that can be taken. In short, the aim of OptimizeU is decrease wait times of
passengers and maximize profits of Uber drivers.

\newpage

\section{Design Description}
\subsection{Module Description and Decomposition}
\subsubsection{Module Overview}
The table below shows a top-level view of each class in the application. Classes are grouped
by their functionality and dependency on one another.
\begin{table}[h]
\begin{tabularx}{\textwidth}{|l|X|X|}
\hline
Function & Class Name & Description\\
\hline
Basic Representation & Cord & 2-Dimensional representation of coordinates\\
\hline
Clustering & Cluster & Representation of a cluster produced by K-Means\\
\cline{2-3}
~ & KMeans & Clustering algorithm API\\
\hline
Minimum Spanning Tree & KruskalMST & Driver class to calculate Minimum
                                     Spanning Tree\\
\cline{2-3}
~ & Edge & ADT representation of weighted edges\\
\cline{2-3}
~ & Graph & Module to instantiate a graph with directed edges\\
\cline{2-3}
~ & UF & Helper module for connecting vertices in a graph\\
\cline{2-3}
~ & Heap & Heapsort on edges for Kruskals\\
\hline
Input/Output & Load & Reads dataset, sorts them according to time,
                      and puts each coordinate in a Hash Table\\
\cline{2-3}
~ & drawSurface & Graphics implementation of clusters and MST\\
\cline{2-3}
~ & demoFrame & Main driver of application\\
\hline
\end{tabularx}
\end{table}

\newpage
\subsubsection{Module Decomposition Semantics}
Following the table in the previous section, the following items below provide a more detailed
description of the rationale behind the modular decomposition of the design.
\begin{itemize}
\item \textbf{Basic Representation:} Since the application makes use of 
coordinates on a two-dimensional plane, the Cord API makes available a 
convenient coordinate ADT that is used whenever a point in the plane needs
to be represented or stored.
\item \textbf{Clustering:} KMeans and its associated ADT, Clusters, are used
to compute clusters from a random number of points. Clusters are stored in the
ADT as a point representing the mean of the cluster, called the centroid.
All points associated to the centroid are stored in the ADT as well
\item \textbf{Minimum Spanning Tree:} Kruskal's algorithm is used to generate
the minimum spanning tree of the clusters' centroids. All the associated modules
are helpers to the algorithm, such as union-find, sorting, and graph/edge
representations.
\item \textbf{Input and Output:} This can be divided by itself into two parts:
input and output. Load is the input routine of the application: It leverages
countsort to sort input data according to the pickup time which is 
partitioned by th 24-hour system. The other two modules are used to collect
output from the algorithms and display it on a graphic canvas.
\end{itemize}

\newpage
\subsubsection{Uses Relation}
An overlay of the application can be viewed in the Uses relationship below:

\newpage
\subsubsection{UML Class Diagram}
An overlay of the application can be viewed in the UML class diagram below:

\newpage
\subsection{Module Interface Specification}
This section will cover:
\begin{itemize}
\item A description of the interface of each public entity, as well as its syntax
and semantics
\item A description of the implementation of private entities of each module,
state variables and how the methods maintain these state variables
\item Requirements trace back for each module
\end{itemize}

\newpage
\subsection{UML State Machine Diagrams}
The diagrams below are UML State Machine Diagrams for KMeans and KruskalMST

\newpage
\section{Internal Review and Evaluation}
The internal review is as follows:
\end{document}